\documentclass[12pt]{article} 

\title{Modelagem em multithread da interação entre colônias de formiga} 
\author{Germano Andrade, João Alcindo e Tiago da Silva} 
\date{\today} 

% language 
\usepackage[portuguese]{babel} 
% general packages 
\usepackage{amssymb} 
\usepackage{amsmath} 
\usepackage{amsthm} 
\usepackage{amsfonts} 
\usepackage{enumitem} 
\usepackage{tabularx} 

% figures 
\usepackage{graphicx} 
\usepackage{tikz} 
\usepackage{tkz-graph} 
\usetikzlibrary{arrows} 
% \usetikzlibrary{calc} 

% theorems environment 
\let \eps \epsilon 
\usepackage{fullpage} 

\newtheorem{theorem}{Teorema} 
\newtheorem{proposition}{Proposição} 

\theoremstyle{definition} 
% \newtheorem{problem}{Exercício} 
\newtheorem{definition}{Definição} 
\newtheorem{lemma}{Lema} 
\newtheorem*{description*}{Problema} 

\SetVertexNormal[Shape      = circle, 
                 LineWidth  = 2pt]
\SetUpEdge[lw         = 1.5pt]

\newcommand\exercisea{\@alpha{a, b, d}}  

\newtheorem{innercustomgeneric}{\customgenericname}
\providecommand{\customgenericname}{}
\newcommand{\newcustomtheorem}[2]{%
  \newenvironment{#1}[1]
  {%
   \renewcommand\customgenericname{#2}%
   \renewcommand\theinnercustomgeneric{##1}%
   \innercustomgeneric
  }
  {\endinnercustomgeneric}
}
\newcustomtheorem{problem}{Exercício}
 
\begin{document}  
	 
\maketitle  
\tableofcontents 

\section{Introdução} 

Neste documento, vamos descrever os aspectos que enformaram nossas implementações e caracterizaram as nossas decisões de modelagem; contemplaremos, além disso, situações tremendamente atribuladas que caracterizaram tanto a implementação serial -- em uma thread -- quanto a multithread deste sistema. Na seção seguinte, portanto, introduzimos os atributos gerais de nosso programa, enfatizando como os agentes -- as formigas, os formigueiros, as comidas e os objetos subjacentes, como os mapas e as coordenadas -- foram desenhados, e apontando, também, para a contemplação dos mecanismos que ensejam sua interação. 

Na seção subsequente, vislumbramos os alicereces que culminaram na versão mulithread do programa, explicitando a utilização de variáveis de exclusão mútua, de variáveis de condição e de semáforos com o objetivo de lograr as idiossincrasias inconvenientes das programação paralela, como as condições de corrida e a inanição (\textit{starvation}). 

\section{Desenho dos agentes} 

\end{document} 
